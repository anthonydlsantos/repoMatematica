\documentclass[11pt, a4paper]{article}
\usepackage{graphicx} % Required for inserting images
\usepackage{amsmath}
\usepackage{titlesec}
\usepackage{blindtext}
\usepackage{hyperref} % Para links, en contenido. 
\usepackage{systeme}
\setcounter{section}{-1}

\hypersetup{
    colorlinks=true,
    linkcolor=blue,
    filecolor=magenta,      
    urlcolor=cyan,
    pdftitle={Overleaf Example},
    pdfpagemode=FullScreen,
    }

\renewcommand*\contentsname{ Contenido :} % Para la tabla de contenido

\title{ L\'imites y continuidad. }
\author{ Anthony de los Santos \footnote{ Los ejercicios y comentarios presentados aqu\'i son de mi responsabilidad, por cualquier
error visto contactar  \textit{agregdelossantos@gmail.com} } }
\date{ 24/10/2023 }

\begin{document}

% ----------- Titulo y contenido

\maketitle 

\newpage

\tableofcontents

\newpage

% ------------


\vspace{20px}

\section{ Sobre estas notas. }

Estas notas estan pensadas para ser una guia en las clases, y tambi\'en ser\'a referencia 
de ejercicios a realizar. 

--- \\
Recomiendo seguir la \textbf{referencia bibliogr\'afica} para un estudio m\'as profundo. Como tambi\'en apoyarse y utilizar herramientas computacionales para realizar gr\'aficas y c\'alculo de l\'imites. Algunos ejemplos de tales herramientas: Geogebra, matem\'atica con Python, Octave, etc... 

--- \\


% Guia Programa Mat 3ero Informatica : 
% - Definir límite finito e infinito.
% - Conocer los teoremas de límites: unicidad, conservación del signo, limite
%    de la función comprendida y limite de la función compuesta.
% - Conocer las operaciones con límites: suma, producto, cociente, potencia
%    y sus casos indeterminados.
% - Resolver ejercicios de límites.
% - Definir funciones equivalentes y conocer sus propiedades.
% - Utilizar las equivalencias fundamentales en la resolución de límites
%    indeterminados.
% - Aplicar los teoremas relativos a los infinitos e infinitésimos en la
%    resolución de problemas.
% - Conocer la definición de continuidad de una función en un punto y en un
%    intervalo.
% - Calcular límites laterales y determinar la existencia del límite de una
%    función en un punto y su continuidad.
% - Clasificar las discontinuidades.
% - Definir extremos absolutos.
% - Aplicar los teoremas de las funciones continuas en un intervalo cerrado:
%    teoremas de: Bolzano, Darboux y Weierstrass.
% - Conocer la demostración del teorema de Darboux
% - Aplicar el teorema de Bolzano en la resolución de ecuaciones por el
%    método de ábacos


\section{ Primeras nociones de 'proximidad'.  }

\subsection{ Noci\'on intuitiva de l\'imite }

Supongamos la funci\'on $ f: \mathbf{R} \to \mathbf{R} $, $ f(x) = x^2 - 4 $. \\  Queremos estudiar el {\color{blue} comportamiento} de la funci\'on $f$ cuando los valores del dominio, los valores de entrada, se aproximan a $1$. Esto ultimo podr\'ia expresarse como {\color{blue} ( $ x \to 1 $ )} que se puede leer como : 
\textit{ el valor x se aproxima hacia 1 ; x tiende hacia 1 }. \\ \\
Para esto, lo que podemos hacer es una tabla con valores $ x | f(x) $. \\
\begin{center}
\begin{tabular}{ | m | m | } 
  \hline
  $ x $ & $ f(x) = x^2 - 4 $  \\ 
  \hline
  $ 0,75 $ & $f(0,75) = -3,43$  \\ 
  \hline
  $ 0,9 $ & $f(0,9) = -3,19  $ \\
  \hline
  $ 0,99 $ & $f(0,99) = -3,019  $ \\
  \hline
  $ 1,001 $ & $f(1,001) = -2,9979  $ \\
  \hline
  $ 1,01 $ & $f(1,01) = -2,9799  $ \\
  \hline
  $ 1,05 $ & $f(1,05) = -2,8975  $ \\
  \hline
\end{tabular}
\end{center}
Con esta tabla, podr\'iamos {\color{blue} conjeturar}, suponer, que cuando x se aproxima hacia 1, {\color{blue} ($ x \to 1$)}, la funci\'on $f(x)$ se aproxima hacia el valor de $-3$, esto lo denotamos como {\color{blue}$ f(x) \to -3 $}. \\ \\ 
Decimos entonces que la funci\'on tiene l\'imite -3 cuando x tiende hacia 1. 

\begin{center}
$ \displaystyle \lim_{x \to 1} f(x) = -3  $
\end{center}

\subsection{ Distancia entre dos puntos en $\mathbf{R}$ }

Tengamos ahora una funci\'on real,  $ f(x) = 2x - 2 $ y conjeturemos el l\'imite de $f(x)$, cuando x tiende hacia 0. $( x \to 0 )$ \\ \\
Por el momento, podemos realizar una tabla de valores, como en la parte anterior, y estudiar el comportamiento de $f(x)$, cuando $x \to 0$\\ \\
Al realizar este estudio podr\'iamos decir que {\color{blue} $f(x) \to -2$ {\color{black}cuando} $x \to 0 $}\\ \\
{\color{magenta} Ejercicio : Realizar la tabla de valores y verificar la conjetura del L\'imite de $f(x)$ }\\ \\ 
Una pregunta que surge, {\color{blue}¿ Que tan pr\'oximo est\'a f(x) de -2 ?} \\ \\
Esto depende de cuanto queramos que sea esa proximidad. Pero para eso, debemos tener una idea de 'medida' en los numeros reales, o mejor dicho, una noci\'on de distancia entre dos puntos. \\
\begin{center}
    Dado dos puntos, $a,b \in \mathbf{R}$ se define la distancia entre a,b como: $d(a,b) = | a - b |$
\end{center}
Supongamos el caso en que deseamos que $f(x)$ diste de -2 {\textit{(su l\'imite)} menos que 1 unidad. Esto es, {\color{blue} $ | f(x) - (-2) | < 1 $} \\ \\ 
Pasamos al desarrollo de cuentas : 
\begin{center}
    $ | f(x) - (-2) | < 1$ \\ \hfill
    
    $ | (2x - 2) - (-2) | < 1$\\ \hfill

    $ | 2x - 2 + 2 | < 1$ \\ \hfill

    $ | 2x | < 1$ \\ \hfill

    $ 2| x | < 1$ \\ \hfill

    $ | x | <  \frac{1}{2} $ \\ \hfill

   {\color{green} $ | x - 0 | <  \frac{1}{2} $ }\\ \hfill

    
\end{center}

Que interpretar de esto? Recordemos que conjeturamos el l\'imite de f(x) cuando x se aproxima hacia 0. Dicho l\'imite es -2. \\ \\ Lo que procuramos es, aquellos valores de $x$ que permiten que $f(x)$ este a menos de 1 unidad del l\'imite -2.\\

Mejor dicho, cuando la distancia entre $x$ y $0$ sea menor que $\frac{1}{2}$, \\
{\color{blue} $| x - 0 | < \frac{1}{2} $}, la funci\'on $f(x)$ estar\'a pr\'oxima a -2, tal distancia entre f(x) y -2 es menor a 1 unidad, {\color{blue}$| f(x) - (-2) | < 1 $} \\ \\ 

Otra cuesti\'on ... Si ahora quiero que f(x) diste de -2, un valor menor que 0,05 unidades. {\color{magenta}¿ Que distancia debe estar x de 0 para que f(x) diste de -2, menos de 0,05 unidades ?  
} \\ \\
Respuesta : $| x - 0 | < 0,025$ \\

\textit{{\color{magenta} Investigar y confirmar estos resultados con la tabla de valores $x|f(x)$. }}
 
\section{ Definiciones formales }
\subsection{Definici\'on de l\'imite}
Siguiendo con la funci\'on y ejemplo anterior, vimos que si deseamos que la distancia de $f(x)$ a su l\'imite $L$ sea menor que 1 unidad, tenemos que x debe estar a una distancia de 0, menor que $1/2$. Esto lo podemos hacer con cualquier valor positivo, \textit{tan peque\~no cuanto quieramos}. La distancia de mi funci\'on $f(x)$ a su l\'imite $L$ tan \textit{pr\'oxima}. Esto si, suponiendo que este l\'imite existe. Pasamos entonces a la definici\'on formal: 

\begin{center}
    La funci\'on $f(x)$ tiene l\'imite $L$ cuando $x$ tiende hacia un valor $a$ si, \\ 
    {\color{blue} $ \forall \epsilon >0, ~ \exists \delta >0 : |x-a|<\delta \implies |f(x) - L|< \epsilon $ }
\end{center}
Esta es la conocida definici\'on $\epsilon-\delta$ \textit{(epsilon-delta)} de L\'imite de una funci\'on en un punto.\\ \\
{\color{green}\textbf{Ejemplo:} Probar que $ \displaystyle \lim_{x \to 0} 2x - 2 = -2 $ } \\ \\ 
Para esto trataremos de encontrar un $\delta$ en funci\'on de $\epsilon$ tal que se cumpla lo de la definici\'on. 
\begin{center}
    {\color{blue}$ |f(x) - L | < \epsilon $} ~ en este caso, $ | (2x - 2) - (-2) | < \epsilon $, con $\epsilon>0 $ \\ \hfill
    
    $ | 2x | < \epsilon $, que lo puedo expresar como, $|2(x-0)| < \epsilon $\\ \hfill

    $ 2{\color{blue}|x-0|} < \epsilon $, por \'ultimo, {\color{blue}$|x-0| < \frac{\epsilon}{2}$} \\ \hfill

    Defino entonces {\color{blue}$\delta = \frac{\epsilon}{2}$}, tal que se cumple lo expresado en la  definici\'on.
\end{center}
Esto es, cuando la distancia de cualquier punto $x$ a 0, sea menor que $\epsilon/2$, $f(x)$ estar\'a tan pr\'oxima a su l\'imite, a una distancia menor que $\epsilon$, sea este valor el que se quiera, tan peque\~no, pero mayor a cero ($\epsilon > 0$). 
\subsection{Operaciones con L\'imites}
Sea $\displaystyle \lim_{ x\to a} f(x) = L $, $\displaystyle \lim_{ x\to a} g(x) = M $ 
\begin{center}
\begin{tabular}{ | m | m | } 
  \hline
  $\displaystyle \lim_{ x\to a} ( f(x) \pm g(x) ) $ & $ L \pm M  $  \\ 
  \hline
  $\displaystyle \lim_{ x\to a}  f(x)g(x)  $ & $LM$  \\ 
  \hline
  $\displaystyle \lim_{ x\to a} kf(x)  $ & $kL$ ~ siendo $k \in R$ \\
  \hline
  $\displaystyle \lim_{ x\to a} \frac{f(x)}{g(x)}$ ~ con $g(x) \neq 0 $ & $\frac{L}{M}  $ \\
  \hline
  $\displaystyle \lim_{ x\to a} \sqrt{f(x)} $ & $\sqrt{L} $ \\
  \hline
  $\displaystyle \lim_{ x\to a} ( f(x) )^n $ & $L^n $ \\  \hline
  
\end{tabular}
\end{center}
{\color{green} \textbf{Ejemplo:} Si $\displaystyle \lim_{ x\to -2} f(x) = 1 $, $\displaystyle \lim_{ x\to -2} g(x) = -1 $, ~ esto es, $L=1$, $M=-1$  } \\ \\
a) $\displaystyle \lim_{ x\to a} ( f(x) + 5g(x) ) = -4$ \\ \\
b) $\displaystyle \lim_{ x\to a}  f(x)g(x) = -1 $ \\ \\
c) $\displaystyle \lim_{ x\to a} \frac{f(x)}{g(x)} = -1 $\\ \\ 
{\color{magenta} Compruebe y explique que propiedades utiliz\'o para el c\'alculo de los l\'imites.} \\ \\
---\\ \\
\textbf{C\'alculo de l\'imites, primeros casos indeterminados $\frac{0}{0}$ }\\ \\
Supongamos que tenemos una funci\'on $f(x) = x^2 + 3x - 5$ y queremos saber su l\'imite cuando x tiende hacia 3\\ 
Lo que podemos hacer es de cierta forma {\color{magenta}'evaluar'} la funci\'on en ese punto, cuidado por que la $f(x)$ no necesita estar definida en ese punto para calcular el l\'imite. \\ \\
$ \displaystyle \lim_{x\to3} ~ x^2 + 3x - 5 = (3)^2 +3(3) - 5 ~=~ 13  $ \\ \\
Decimos que el l\'imite de la funci\'on $f(x)$ es 13 cuando $x \to 3$ \\ \\ 
En general, una 'estrategia' para el c\'alculo de l\'imites es evaluar la funci\'on en el punto de tendencia y estudiar el resultado. \\ \\ 
Supongamos ahora que queremos c\'alcular el l\'imite, { \color{blue} $\displaystyle \lim_{ x\to 3} \frac{x^2-9}{x-3} $} \\ \\
Esta funci\'on No esta definida cuando $ x=3 $ ya que su denominador es nulo cuando x vale 3.\\ 
Por ende tenemos que, {\color{blue} $\displaystyle \lim_{ x\to 3} \frac{x^2-9}{x-3} = \frac{0}{0}$ } \\ \\ 
En este caso decimos que el \textit{l\'imite es indeterminado}. Existen otros tipos de indeterminados, que ya veremos. Cuando esto sucede, hay distintas estrategias a seguir para tratar de resolver esa indeterminaci\'on. \\ \\
En este caso, basta con un poco de \'algebra para resolver el problema.\\ 
\begin{center}
    Primero veamos que {\color{blue} $x^2 - 9 = (x-3)(x+3)$} \\ \hfill
    
    por ende, se puede escribir el l\'imite como {\color{blue} $\displaystyle \lim_{ x\to 3} \frac{(x-3)(x+3)}{x-3} $ } \\ \hfill 

    Si evaluamos el l\'imite seguir\'a indeterminado, pero este l\'imite se puede simplificar, cancelando los terminos $(x-3)$ que estan en el numerador y en el denominador.\\ \hfill

    {\color{blue} $\displaystyle \lim_{ x\to 3} \frac{(x-3)(x+3)}{x-3} = \lim_{ x\to 3} x + 3 = 6 $ } \\ \hfill 

    por ende, {\color{blue} $\displaystyle \lim_{ x\to 3} \frac{x^2-9}{x-3} = 6 $ }
\end{center}
\newpage
\section{ Lista de Ejercicios - L\'imites - Aproximaciones y operaciones b\'asicas }

\textbf{Ejercicio 1)  Aproximaciones } \\ \\
Aproximar el l\'imite de las siguientes funciones utilizando una tabla de valores. Pueden ayudarse con calculadoras o distintos programas de c\'alculo. \\ \\
a) $ \displaystyle  \lim_{x \to 0} \frac{ \sqrt{x+4} - 2 }{ x } $ \\ \\
b) $ \displaystyle  \lim_{x \to 0} \frac{ 9^x - 5^x }{ x } $ \\ \\
c) $ \displaystyle  \lim_{x \to 1} \frac{ x^6 - 1 }{ x^{10} - 1 } $ \\ \\

\textbf{ Ejercicio 2) Operaciones con l\'imites } \\ \\ 
Teniendo que, $ \displaystyle  \lim_{x \to 2} f(x) = 4  $, $ \displaystyle  \lim_{x \to 2} g(x) = -2  $. Resolver \\ \\ 
a) $ \displaystyle  \lim_{x \to 2} ( f(x) + 5g(x) )  $ \\ \\ 
b) $ \displaystyle  \lim_{x \to 2} \sqrt{ f(x) }  $ \\ 

\textbf{ Ejercicio 3) Prueba de l\'imite utilizando la definici\'on } \\ \\
Probar que $ \displaystyle  \lim_{x \to 10}  3 - \frac{4}{5}x = -5 $ \\ 

\textbf{ Ejercicio 4) C\'alculo de l\'imites } \\ \\ 
a) $ \displaystyle  \lim_{x \to 5} \frac{x^2 - 6x + 5}{x-5} $ \\ \\ 
b) $ \displaystyle  \lim_{x \to 2} \frac{x+2}{x^3 + 8}  $ \\ 

\textbf{ Ejercicio 5) Investigar } \\ \\ 
Dado el siguiente l\'imite, discutir sobre su valor. \\ 
a) $ \displaystyle  \lim_{x \to 5} \frac{x^2 - 5x + 6}{x-5} $ \\ 
b) $ \displaystyle  \lim_{x \to 0} \frac{1}{x^2}  $

\section{Limites laterales, Limite infinito}

Cuando queremos saber el L\'imite de una funci\'on cuando los valores de {\color{blue} $x$} tienden hacia un valor {\color{blue} $a$ }, lo que hacemos es 'aproximarnos' del punto {\color{blue} $a$}, tomando valores menores y mayores  que este, muy pr\'oximos al valor mismo de {\color{blue} $a$}. Esto es, si tengo el caso en que {\color{blue} $x \to 2$ } entonces puedo tomar los siguientes valores que se aproximan hacia 2 : 

\begin{center}
    $ 1,80 ~|~ 1,90 ~|~ 1,99 ~|~ 1,9999 ~|~ 2,0001 ~|~ 2,001 ~|~ 2,01 ~|~ 2,05 $    
\end{center}
Supongamos que ahora quiero estudiar el caso, cuando me aproximo hacia 2, pero con aquellos valores mayores a 2. En este caso decimos que me aproximo hacia 2 {\color{blue} por la derecha } y lo denotamos {\color{blue} $x\to 2^+$}\\ \\
Estos valores podr\'ian ser : $ 2,0001 ~|~ 2,001 ~|~ 2,01 ~|~ 2,05 $ \\ \\     
Caso an\'alogo para aproximarme con valores menores a 2. Decimos que me aproximo hacia 2, por {\color{blue} la izquieda} y lo denotamos {\color{blue} $x\to 2^-$} \\ \\
\textbf{Definici\'on de L\'imites laterales} \\ \\ 
El L\'imite de la funci\'on {\color{blue} $f(x)$} cuando $\color{blue} x $ tiende hacia {\color{blue} a} por la derecha existe y es igual a {\color{blue} $L$}, {\color{blue}$ \displaystyle \lim_{x \to a^+} f(x) = L  $} ~ si tenemos que, 
\begin{center}
    {\color{blue} $ \forall \epsilon > 0, \exists \delta >0 ~:~ a < x < a + \delta  \implies | f(x) - L | < \epsilon $}
\end{center}
{\color{magenta} Ejercicio : Escribir la definici\'on de L\'imite de $f(x)$ cuando $x$ tiende hacia el punto $a$, por la izquierda;  $ \displaystyle \lim_{x \to a^-} f(x) = L  $ } \\ \\
{\color{green} \textbf{Ejemplos: } } \\ \\ 
a) Calcular el l\'imite por derecha e izquierda (l\'imites laterales) cuando x tiende hacia 0, de la siguiente funci\'on, \\ \\
$ 
    f(x) = \begin{cases}
              1 & \text{si } x \geq 0,\\
               0 & \text{si } x < 0.
          \end{cases}
  $ ~ ~ $ \displaystyle \lim_{x \to 0^+} f(x) = 1 $ ~ ~ $ \displaystyle \lim_{x \to 0^-} f(x) = 0 $ \\ \\ \\
b) Calcular l\'imites laterales, cuando x tiende hacia 1 de la siguiente funci\'on g(x), \\ \\
$ 
    g(x) = \begin{cases}
              x^2 & \text{si } x \neq 1,\\
               2 & \text{si } x = 0.
          \end{cases}
  $ ~ ~ $ \displaystyle \lim_{x \to 1^+} g(x) = 1 $ ~ ~ $ \displaystyle \lim_{x \to 1^-} g(x) = 1 $ \\ \\ \\
c) Calcular $ \displaystyle \lim_{x \to -4} |x+4| $ ~ $( h(x) = |x+4| ) \to 0 $ cuando $( x \to -4)$ \\ {\color{magenta} Ejercicio: Comprobar} \\ \\

{\color{magenta} Ejercicio: Calcular los l\'imites anteriores, pero teniendo en cuenta su gr\'afico. Esto es, sin c\'alculos, solamente viendo la gr\'afica de la funci\'on. } \\
\begin{center}
    \textbf{Concepto Importante!!!} \\ 
    El l\'imite de una funci\'on existe, s\'i y solamente si, los l\'imites laterales existen y son ambos iguales. \\ \hfill
    
    {\color{blue}$ \displaystyle \lim_{x \to a} f(x) = L  $ ~sii~ $ \displaystyle \lim_{x \to a^+} f(x) = \displaystyle \lim_{x \to a^-} f(x) = L
    $} 
\end{center}
\textbf{L\'imites en el infinito} \\ \\
Tenemos del Ejercicio 5.b) de la Lista de ejercicios, que si $(x \to 0^+)$,\\ $1/x^2$ toma valores muy grandes, lo mismo sucede cuando $(x \to 0^-)$ \\
En este caso decimos que {\color{blue} ($ f(x) = 1/x^2 \to \infty $) }, la funci\'on tiende hacia el infinito. \\

{\color{magenta} Ejercicio: Investigar que pasa con $log(x)$ cuando $x\to 0^+$} 

\begin{center}
    \textbf{Definici\'on : L\'imite infinito cuando $x\to a$} \\ \hfill

    Decimos que el l\'imite de una funci\'on $f(x)$ tiende hacia el infinito, cuando $ x\to a $ si para cualquier valor $M>0$, se tiene \\ \hfill

    {\color{blue} $| x-a | < \delta \implies f(x) > M $}
\end{center}

{\color{magenta} Ejercicio: Completar la definici\'on cuando $( f(x) \to -\infty )$} \\ \\ 
Usamos la notaci\'on {\color{blue}$ \displaystyle \lim_{x \to a} f(x) = \infty $} En estos casos se dice tambi\'en que la \textit{funci\'on $f(x)$ es un infinito en cierto punto $a$. }


\newpage
\section{ Lista de Ejercicios: C\'alculo de L\'imites, L\'imite infinito }
\textbf{ Ejercicio 6) C\'alculo de l\'imites - 2da parte } \\ \\
a) $ \displaystyle  \lim_{x \to 3} \frac{ 2x^3 + 2x^2 -16x -24 }{ x - 3} $ \\ \\
b) $ \displaystyle  \lim_{x \to 0} \frac{ \sqrt{x+4} - 2 }{ x } $ \\ \\
c) $ 
    f(x) = \begin{cases}
              2x + 10 & \text{si } x \leq -2,\\
               -4x + x & \text{si } x > -2.
          \end{cases}
  $ ~ Hallar el l\'imite de $f(x)$, cuando $ x \to -2 $ \\ \\
d) $ \displaystyle  \lim_{x \to 3} \frac{ |x-3| }{ x-3 } $ \\ \\ 
e) $ \displaystyle  \lim_{x \to 1} \frac{ x^2 - 1 }{ \sqrt{x}-1 } $ \\ \\
\textbf{ Ejercicio 7) }\\ \\
Calcular los siguientes l\'imites \\ \\ 
a) $ \displaystyle \lim_{x \to 3^+} log(x-3)  $ \\ \\
b) $ \displaystyle \lim_{ x \to 1^+} \frac{1}{x^2 - 1} $ \\ \\
c) $ \displaystyle \lim_{ x \to -1^-} \frac{1}{x^2 - 1} $ \\ \\
d) $ \displaystyle \lim_{ x \to -2^+} \frac{log(x+2)}{3} $ \\ \\

\newpage
\section{Infinitos, infinit\'esimos}

\subsection{ L\'imites y as\'intotas }

Hasta el momento, estudiamos el comportamiento de una funci\'on {\color{blue} $f(x)$} cuando la variable {\color{blue}$x$} se aproxima hacia un determinado valor {\color{blue} $a$}, ($ x \to a $).\\ \\ 
Desde ahora, consideramos el caso en el que $x$ toma valores arbitatrios, y suficientemente grande. Lo notaremos como {\color{blue}( $ x \to \infty $ )}\\
Tambi\'en se considera el caso an\'alogo, cuando x toma valores muy peque\~nos, o negativos, en este caso lo notamos {\color{blue}($x \to - \infty$)} \\ \\ 
Como el caso anterior diremos que la funci\'on $f(x)$ tiene l\'imite $L$ cuando la variable $x$ tiende hacia el infinito {\color{blue}( $ x \to \infty $ )} si, \\
\begin{center}
    {\color{blue} $\forall \epsilon > 0, \exists N : |x| > N \implies | f(x) - L | < \epsilon $ } \\ \hfill
    
    Notaci\'on: {\color{blue} $ \displaystyle \lim_{ x \to \infty } f(x) = L   $}
\end{center}

{\color{magenta} Ejercicio : Completar la definici\'on para cuando $( x \to - \infty )$ }\\ \\
Comencemos estudiando el comportamiento de la funci\'on $ \displaystyle f(x) = \frac{x^2-1}{x^2+1} $ cuando $x \to \infty $ \textit{{\color{magenta} Ejercicio : Gr\'aficar la funci\'on con ayuda de alguna herramienta computacional, ej: Geogebra, Python, etc... }} \\
\begin{center}
    Planteo de cuentas: \\ \hfill
    
    $\displaystyle \lim_{x \to \infty} \frac{x^2-1}{x^2+1}  $ ~ Tendiendo que podemos escribir {\color{blue} $\displaystyle x^2 - 1 = x^2(1-\frac{1}{x^2})$} y lo mismo para $\displaystyle x^2 + 1 = x^2( 1 + \frac{1}{x^2} )$ El l\'imite lo expresamos como: \\ \hfill

    $\displaystyle \lim_{x \to \infty} \frac{x^2-1}{x^2+1} = \lim_{x\to\infty} \frac{x^2(1-\frac{1}{x^2})}{x^2(1+\frac{1}{x^2})} = \lim_{x\to\infty} \frac{1-\frac{1}{x^2}}{1+\frac{1}{x^2}} $ \\ \hfill

    Cuando {\color{blue} $x \to \infty $} los terminos {\color{blue}$\frac{1}{x^2}$} tienden hacia {\color{blue}$0$} \\ \hfill

    Por ende, $\displaystyle \lim_{x \to \infty} \frac{x^2-1}{x^2+1} = 1 $     
\end{center}

El hecho de que $f(x)$ tiende hacia $1$ cuando $x \to \infty$ tiene un signficado gr\'afico tambi\'en. La gr\'afica de la funci\'on $f(x)$ se aproxima a la recta $y=1$ cuando x toma valores suficientemente grandes. Decimos entonces que $f(x)$ tiene una {\color{blue} As\'intota horizontal} $y=1$, cuando $x\to \infty$ \\ 

\begin{center}
    Definici\'on: Decimos que $f(x)$ tiene una {\color{blue} as\'intota horizontal, $y = L$ } cuando: \\ \hfill

    $ \displaystyle \lim_{x\to\infty} f(x) = L $ ~;~ $ \displaystyle \lim_{x\to -\infty} f(x) = L $ 
\end{center}

---\\ 
El caso en que $ \displaystyle \lim_{x\to a^+} f(x) = \infty $ ~;~ $ \displaystyle \lim_{x\to a^-} f(x) = -\infty $ decimos que en la recta $x=a$ la funci\'on $f(x)$ tiene una {\color{blue} as\'intota vertical, de ecuaci\'on $x=a$ } {\color{magenta} Ejemplo: Investigar asintota vertical de $g(x)= \frac{1}{x+2} $ cuando $ x \to -2^{\pm} $ } \\ \\

\subsection{Ordenes de infinitos e infinit\'esimos }
{\color{magenta} Ejercicio: Realizar y comparar en una tabla de valores las funciones $e^x$, $x$, $log(x)$, cuando $x\to \infty $   } \\ \\ 

\begin{center}
    \textbf{Definiciones:} Decimos que una funci\'on $f(x)$ es un {\color{blue} infinit\'esimo } cuando $x \to a$ si $ \displaystyle \lim_{x\to a} f(x) = 0 $ \\ 
    Decimos que $f(x)$ es un {\color{blue}infinito} si $ \displaystyle \lim_{x\to a} f(x) = \infty $
\end{center}

Funciones comparables : $f$ y $g$ ambas funciones, decimos que son comparables si $ \displaystyle \lim_{x\to a} \frac{f(x)}{g(x)} $, $ \displaystyle \lim_{x\to a} \frac{g(x)}{f(x)} $  existe en los Reales o son infinitos. \\ \\ 
\textit{{\color{red} Discutir e introducir ordenes de infinitios e infit\'esimos en clase ... }} \\ \\ 
Decimos que dos infinit\'esimos o infinitos $f$ y $g$ son {\color{blue} equivalentes} cuando $x \to a$, y lo anotamos 
{\color{blue} $ f(x) \sim  g(x)  $} si {\color{blue}$ \displaystyle \lim_{x\to a} \frac{f(x)}{g(x)} = 1 $} \\ 
\begin{center}
    Sean $f$ y $g$ infinit\'esimos o infinitos cuando $x \to a $. Supongamos que existe el l\'imite: $ \displaystyle \lim_{x\to a} \frac{f(x)}{g(x)} $ y tenemos tambi\'en que $ f(x) \sim h(x) $ \\ \hfill

    entonces, {\color{blue}$ \displaystyle \lim_{x\to a} \frac{f(x)}{g(x)} =  \lim_{x\to a} \frac{h(x)}{g(x)}  $} \\ \hfill

    An\'alogamente, si existe $ \displaystyle \lim_{ x\to a} f(x).g(x) $, tenemos que, \\ {\color{blue} $ \displaystyle \lim_{ x\to a} f(x).g(x) = \lim_{ x\to a} h(x).g(x) $ } \\ \hfill

    
\end{center}
Esta herramienta es \'util a la hora de calcular l\'imites que parecen complejos pero si conocemos algunos infinitos o infinit\'esimos equivalentes puede que se simplifiquen los c\'alculos. \\ \\

{\color{green} Ejemplo : } \\ \\
Supongamos que deseamos calcular el l\'imite {\color{blue}$ \displaystyle \lim_{ x \to 0 } \frac{e^{x} - 1}{x} $} \\ 
Si hacemos tender x hacia 0, llegamos a una indeterminaci\'on del tipo 0/0. \\
Si bien hay distintas herramientas para resolver casos indeterminados, este caso ya es un tanto m\'as complejo que los anteriores. \\ \\ La idea en este caso es tratar de usar {\color{blue} equivalentes} para resolver dicho l\'imite. \\
Por un lado, tenemos la equivalencia {\color{blue}$ e^{x} - 1 \sim x $} cuando {\color{blue}$x \to 0$} \\ 

Entonces, este l\'imite puede expresarse como, \\
\begin{center}
    {\color{blue}$ \displaystyle \lim_{ x \to 0 } \frac{e^{x} - 1}{x} =  \lim_{ x \to 0 } \frac{x}{x} = 1$ }  \\ \hfill

    
\end{center}
Claro que este l\'imite se pudo resolver gracias a que ya se conoc\'ia la equivalencia $ e^{x} - 1 \sim x $ cuando x tiende hacia 0. Una idea de esta equivalencia se podr\'ia entender cuando se estudia Derivadas y aproximaciones de funciones por polinomios \textit{(ej: Desarrollo de Taylor)}. \\ \\ 
Suele ser \'util recurrir a una Tabla de infinit\'esimos equivalentes e infinitos equivalentes, para trabajar con algunos l\'imites. Existen equivalencias b\'asicas, como la  usada en este ejemplo. \\ \\ 
\textit{{\color{magenta} Podr\'ia estudiarse el comportamiento de la funci\'on $f(x) = e^{x} - 1$ en un entorno del 0 de un modo gr\'afico y comparar con la funci\'on $g(x) = x $ en tal entorno. }}


\newpage
\section{Lista de Ejercicios - Infinitos y 
As\'intotas}
\textbf{Ejercicio 8) Infinitos y as\'intotas } \\
Calcular y hallar, si existen, los l\'imites y as\'intotas siguientes \\ \\ 
a)$ \displaystyle \lim_{x \to \infty} \frac{x^3 + 6x^2-x+8}{x^2+5x-1}  $ \\ \\ 
b)$ \displaystyle \lim_{x \to \infty} \frac{5x^2 + 25}{10x^3+7x^2+x}$ \\ \\
c)$ \displaystyle \lim_{x \to -\infty} \frac{1 + x^6}{ x^4 + 1} $ \\ \\ 
d)$ \displaystyle \lim_{x \to \infty} \sqrt{9x^2+x} -x $ \\ \\
\textbf{Ejercicio 9) Ordenes, infinit\'esimos }\\ 
a) Investigar sobre infinit\'esimos e infinitos equivalentes b\'asicos. \textit{(La idea es conocer algunos equivalentes b\'asicos para trabajar despu\'es con l\'imites.)} \\ \\ 
b) A partir de los equivalentes conocidos en la parte a), calcular los siguientes l\'imites: \\

1) $ \displaystyle \lim_{x \to 0} \frac{sin(x)(x+1)}{e^{x}-1} $ \\

2) $ \displaystyle \lim_{x \to 0} \frac{1-cos(x)}{x^2} $ \\

3) $ \displaystyle \lim_{x \to 0} \frac{x^2}{log(1+x)} $ \\

\newpage

\section{Continuidad}
Cuando estamos calculando l\'imites de una funci\'on $f(x)$, estudiamos el comportamiento de esa funci\'on cuando $x$ se aproxima a cierto valor $a$, esto es, $x \to a$. No interesa, para nuestra definici\'on de l\'imite ni para nuestro estudio, cual es el valor de la funci\'on en el punto $a$, esto es, la imagen del punto $a$, $f(a)$. La funci\'on puede que No este definida en ese punto. \\ \\
Supongamos ahora el caso en que Si interesa tener en cuenta este valor, la imagen de $a$, $f(a)$, cuando estamos calculando el l\'imite de una funci\'on $f(x)$. Con esto queremos decir la siguiente definici\'on, \\
\begin{center}
    Decimos que una funci\'on $f$ es {\color{blue}continua} en el punto $a$ cuando, \\ \hfill

    {\color{blue}$ \displaystyle \lim_{x \to a} f(x) = f(a)  $} \\ \hfill

    Definici\'on formal: \\ \hfill

    $\forall \epsilon > 0, \exists \delta > 0 : |x-a|<\delta ~ \implies ~ |f(x) - f(a)|< \epsilon  $
\end{center}

Una funci\'on que es continua en todos sus puntos, se dice continua. \\
Cabe destacar en la definici\'on que para que cierto l\'imite exista entonces sus l\'imites laterales deben ser iguales, esto es l\'imite cuando $x \to a^+$, $x \to a^-$, estos l\'imites iguales al valor $f(a)$. \\ \\ 
Un ejemplo b\'asico de funciones continuas son las funciones polin\'omicas, ejemplo de estas, $f(x) = x^3 + 2x - 1$. \\ \\ 
{\color{magenta} Ejercicios: 
Probar que las siguientes funciones son continuas en los puntos determinados. } \\ \\ 
1) $\displaystyle f(x) = \frac{1}{e^{2x}}$,  $x=2$ \\ \\
2) $g(x) = 2x + 1 - log(x) $, $x=5$ \\ \\ 
\textbf{Puntos en donde la funci\'on No es continua} \\ 
En algunos casos, suele ser de interes saber en donde una funci\'on No es continua, {\color{blue}(discontinua)}. Estos puntos pueden ser en donde la funci\'on No esta definida. 
\newpage
{\color{green} Ejemplos : } \\ \\ 
Supongamos como primer caso la funci\'on $ \displaystyle f(x)= \frac{x^2-x-2}{x-2}$ \\ \\ 
esta funci\'on No esta definida cuando $x = 2$, por ende no existe $f(2)$. Por lo tanto esta funci\'on No es continua en $x=2$. \\ \\ 
Otro ejemplo, sea la funci\'on $ \displaystyle
    g(x) = \begin{cases}
              \frac{1}{x^2} & \text{si } x \neq 0,\\
               1 & \text{si } x = 0.
          \end{cases}
  $ \\ \\ 
Para saber si $g(x)$ es continua en $x=0$ lo que podemos hacer es calcular los l\'imites laterales de $g(x)$ cuando $x \to 0^{-}$, $x \to 0^{+}$ \\ \\ 
Ya sabemos por \textit{ejercicio 5.b)} que $\displaystyle \lim_{x \to 0^{+}} f(x) = \infty $. Lo mismo para el caso en que $x \to 0^{-}$. Tenemos que el L\'imite no existe cuando $x \to 0$. Por m\'as que $f(0) = 1$, tenemos que la funci\'on $g(x)$ No es continua en $x = 0$  \\ \\ 

{\color{magenta} Ejercicio: Discutir continuidad de la funci\'on $g(x) = | x - 4 |$ en $x = 4$. Graficar la funci\'on } \\ 

\begin{center}
    \textbf{Definiciones} \\ 
    Decimos que una funci\'on $f(x)$ es {\color{blue}continua por la derecha} de un punto $x=a$ si, \\ \hfill

    {\color{blue} $\displaystyle \lim_{x \to a^{+}} f(x) = f(a) $} \\ \hfill
    
    Caso an\'alogo para continuidad por la izquierda. \\ \hfill

    Decimos que una funci\'on $f(x)$ es {\color{blue} continua sobre un intervalo} si es continua en cada uno de sus puntos en tal intervalo  
    
\end{center}

\newpage

\section{Lista de ejercicios: Continuidad }
\textbf{Ejercicio 10)} \\
Estudiar la continuidad de las siguientes funciones en los puntos determinados.  \\ \\
a) $\displaystyle f(x) = \frac{1}{x+2} $ ~ ~ en $x = -2 $ \\ \\ 
b) $ g(x) = \begin{cases}
              e^x & \text{si } x < 0,\\
               x^2 & \text{si } x \geq 0.
          \end{cases} $ ~ ~ en $x=0$ \\ \\
c) $ h(x) = \begin{cases}
              cos(x) & \text{si } x < 0,\\
               0 & \text{si } x = 0,\\
               1-x^2 & \text{si } x > 0.
          \end{cases} $ ~ ~ en $x=0$ \\ \\
d) $ \displaystyle i(x) = \frac{2x^2 - x - 1}{x^2 + 1} $ ~ ~ en $x=-1$ \\ \\ 
\textbf{Ejercicio 11)} \\ 
Probar si las siguientes funciones son continuas en sus intervalos determinados, \\ \\ 
a) $ \displaystyle f(x) = \frac{2x + 3}{x-2} $ ~ ~ $I = ( 2, \infty )$ \\ \\
b) $ \displaystyle g(x) = 5\sqrt{3-x} $ ~ ~ $I=(-\infty,3)$ \\ \\
\textbf{Ejercicio 12) } \\ 
Encontrar los valores de ciertos par\'ametros para que la funci\'on sea continua. \\ \\
a) $ f(x) = \begin{cases}
              log(x+1) & \text{si } x > 0,\\
               (x+m)^2 & \text{si } x \leq 0.
          \end{cases} $ ~ ~ en $x=0$\\ \\ \\
b) $ g(x) = \begin{cases}
              x^2 + 3x + 2 & \text{si } x \leq 1,\\
               mx^2 + nx +1 & \text{si } x > 1.
          \end{cases} $ ~ ~ en $x=1$  

\newpage

\section{Teoremas de funciones continuas} 
{ \color{red} En construcci\'on ... } 

\newpage 

\section{Lista de ejercicios - Teoremas de funciones continuas }
\textbf{Ejercicio 12)} \\
{ \color{red} En construcci\'on ... }
% --- 
\newpage

\section{ Referencia bibliogr\'afica }

\begin{itemize}
    \item James Stewart - C\'alculo de una variable. Trascendentes tempranas 7ma Ed. 
    \item  Stewart et al (2012). Prec\'alculo. Matem\'aticas para el c\'alculo 6ta Ed.
    
\end{itemize}

---

\section{ Apendice A: Ejercicios de \'algebra. Operaciones b\'asicas.   }
    
\textbf{ Ejercicio 1) } 
Resolver las siguientes operaciones: \\ \\
a) $ (x-1)^3 - (x+2)^2 $ \\ \\
b) $  \displaystyle  \frac{x^2 + 3x + 2}{ x^2 - x - 2 } $ \\ \\
c) $  \displaystyle  \frac{x^2}{ x^2 - 4 } - \frac{x+1}{x+2} $ \\ 

\textbf{ Ejercicio 2) } \\ \\
Resuelva : \\ \\
a) $ 3(x+6) + 4(2x-5) $ \\ \\
b) $ (2x + 3)^2 $ \\

\textbf{ Ejercicio 3) } \\ \\
Realizar un tabla de valores de las siguientes funciones: \textit{(por ejemplo, pruebe con valores del intervalo (-2,2) ) } \\ \\
a) $ f(x) = 2x + 4 $ \\
b) $ g(x) = e^{-2x} $ \\
c) $ \displaystyle h(x) = \frac{1}{x-1} $ \\
d) $ i(x) = log( x + 10) $ \\ 

\textbf{ Ejercicio 4) } \\ \\
Comportamiento de una funci\'on $f(x)$ cuando $x$ se aproxima hacia un determinado valor. \\

a) $ f(x) = 2x + 5 $ ; cuando $x$ se aproxima hacia $2$ \\ 
Esto ultimo lo podemos escribir como  $( x \to 2 )$. - se dice $x$ tiende a 2 -  





\section{ Apendice B: Funciones por partes, valor absoluto y distancia }

\textbf{Ejercicio 1) Desigualdades con valor absoluto y distancias} \\ \\ 
a) $ | x - 2 | < 5 $ \\ \\
b) $ | 2x + 7 | \leq 2 $ \\ \\
c) $ | x - 2 | > | x + 2 | $ \\ \\
d) $ | 3x + 3 | < | 4( x + 1 ) | $ \\ \\
\textbf{Ejercicio 2) Funciones por partes} \\ \\
Graficar las siguientes funciones.  \\ \\
a) $ f(x) = | 2x + 4 | $ \\ \\
b) $ 
    g(x) = \begin{cases}
              1 & \text{si } x \geq 0,\\
               0 & \text{si } x < 0.
          \end{cases}
  $ \\ \\ \\
c) $ 
    h(x) = \begin{cases}
              x^2 & \text{si } x \neq 1,\\
               2 & \text{si } x = 1
          \end{cases}
  $ \\ \\ \\
d) $ i(x) = | x + 4 | $ \\ \\ 


 
\end{document}
